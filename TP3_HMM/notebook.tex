
% Default to the notebook output style

    


% Inherit from the specified cell style.




    
\documentclass[11pt]{article}

    
    
    \usepackage[T1]{fontenc}
    % Nicer default font (+ math font) than Computer Modern for most use cases
    \usepackage{mathpazo}

    % Basic figure setup, for now with no caption control since it's done
    % automatically by Pandoc (which extracts ![](path) syntax from Markdown).
    \usepackage{graphicx}
    % We will generate all images so they have a width \maxwidth. This means
    % that they will get their normal width if they fit onto the page, but
    % are scaled down if they would overflow the margins.
    \makeatletter
    \def\maxwidth{\ifdim\Gin@nat@width>\linewidth\linewidth
    \else\Gin@nat@width\fi}
    \makeatother
    \let\Oldincludegraphics\includegraphics
    % Set max figure width to be 80% of text width, for now hardcoded.
    \renewcommand{\includegraphics}[1]{\Oldincludegraphics[width=.8\maxwidth]{#1}}
    % Ensure that by default, figures have no caption (until we provide a
    % proper Figure object with a Caption API and a way to capture that
    % in the conversion process - todo).
    \usepackage{caption}
    \DeclareCaptionLabelFormat{nolabel}{}
    \captionsetup{labelformat=nolabel}

    \usepackage{adjustbox} % Used to constrain images to a maximum size 
    \usepackage{xcolor} % Allow colors to be defined
    \usepackage{enumerate} % Needed for markdown enumerations to work
    \usepackage{geometry} % Used to adjust the document margins
    \usepackage{amsmath} % Equations
    \usepackage{amssymb} % Equations
    \usepackage{textcomp} % defines textquotesingle
    % Hack from http://tex.stackexchange.com/a/47451/13684:
    \AtBeginDocument{%
        \def\PYZsq{\textquotesingle}% Upright quotes in Pygmentized code
    }
    \usepackage{upquote} % Upright quotes for verbatim code
    \usepackage{eurosym} % defines \euro
    \usepackage[mathletters]{ucs} % Extended unicode (utf-8) support
    \usepackage[utf8x]{inputenc} % Allow utf-8 characters in the tex document
    \usepackage{fancyvrb} % verbatim replacement that allows latex
    \usepackage{grffile} % extends the file name processing of package graphics 
                         % to support a larger range 
    % The hyperref package gives us a pdf with properly built
    % internal navigation ('pdf bookmarks' for the table of contents,
    % internal cross-reference links, web links for URLs, etc.)
    \usepackage{hyperref}
    \usepackage{longtable} % longtable support required by pandoc >1.10
    \usepackage{booktabs}  % table support for pandoc > 1.12.2
    \usepackage[inline]{enumitem} % IRkernel/repr support (it uses the enumerate* environment)
    \usepackage[normalem]{ulem} % ulem is needed to support strikethroughs (\sout)
                                % normalem makes italics be italics, not underlines
    

    
    
    % Colors for the hyperref package
    \definecolor{urlcolor}{rgb}{0,.145,.698}
    \definecolor{linkcolor}{rgb}{.71,0.21,0.01}
    \definecolor{citecolor}{rgb}{.12,.54,.11}

    % ANSI colors
    \definecolor{ansi-black}{HTML}{3E424D}
    \definecolor{ansi-black-intense}{HTML}{282C36}
    \definecolor{ansi-red}{HTML}{E75C58}
    \definecolor{ansi-red-intense}{HTML}{B22B31}
    \definecolor{ansi-green}{HTML}{00A250}
    \definecolor{ansi-green-intense}{HTML}{007427}
    \definecolor{ansi-yellow}{HTML}{DDB62B}
    \definecolor{ansi-yellow-intense}{HTML}{B27D12}
    \definecolor{ansi-blue}{HTML}{208FFB}
    \definecolor{ansi-blue-intense}{HTML}{0065CA}
    \definecolor{ansi-magenta}{HTML}{D160C4}
    \definecolor{ansi-magenta-intense}{HTML}{A03196}
    \definecolor{ansi-cyan}{HTML}{60C6C8}
    \definecolor{ansi-cyan-intense}{HTML}{258F8F}
    \definecolor{ansi-white}{HTML}{C5C1B4}
    \definecolor{ansi-white-intense}{HTML}{A1A6B2}

    % commands and environments needed by pandoc snippets
    % extracted from the output of `pandoc -s`
    \providecommand{\tightlist}{%
      \setlength{\itemsep}{0pt}\setlength{\parskip}{0pt}}
    \DefineVerbatimEnvironment{Highlighting}{Verbatim}{commandchars=\\\{\}}
    % Add ',fontsize=\small' for more characters per line
    \newenvironment{Shaded}{}{}
    \newcommand{\KeywordTok}[1]{\textcolor[rgb]{0.00,0.44,0.13}{\textbf{{#1}}}}
    \newcommand{\DataTypeTok}[1]{\textcolor[rgb]{0.56,0.13,0.00}{{#1}}}
    \newcommand{\DecValTok}[1]{\textcolor[rgb]{0.25,0.63,0.44}{{#1}}}
    \newcommand{\BaseNTok}[1]{\textcolor[rgb]{0.25,0.63,0.44}{{#1}}}
    \newcommand{\FloatTok}[1]{\textcolor[rgb]{0.25,0.63,0.44}{{#1}}}
    \newcommand{\CharTok}[1]{\textcolor[rgb]{0.25,0.44,0.63}{{#1}}}
    \newcommand{\StringTok}[1]{\textcolor[rgb]{0.25,0.44,0.63}{{#1}}}
    \newcommand{\CommentTok}[1]{\textcolor[rgb]{0.38,0.63,0.69}{\textit{{#1}}}}
    \newcommand{\OtherTok}[1]{\textcolor[rgb]{0.00,0.44,0.13}{{#1}}}
    \newcommand{\AlertTok}[1]{\textcolor[rgb]{1.00,0.00,0.00}{\textbf{{#1}}}}
    \newcommand{\FunctionTok}[1]{\textcolor[rgb]{0.02,0.16,0.49}{{#1}}}
    \newcommand{\RegionMarkerTok}[1]{{#1}}
    \newcommand{\ErrorTok}[1]{\textcolor[rgb]{1.00,0.00,0.00}{\textbf{{#1}}}}
    \newcommand{\NormalTok}[1]{{#1}}
    
    % Additional commands for more recent versions of Pandoc
    \newcommand{\ConstantTok}[1]{\textcolor[rgb]{0.53,0.00,0.00}{{#1}}}
    \newcommand{\SpecialCharTok}[1]{\textcolor[rgb]{0.25,0.44,0.63}{{#1}}}
    \newcommand{\VerbatimStringTok}[1]{\textcolor[rgb]{0.25,0.44,0.63}{{#1}}}
    \newcommand{\SpecialStringTok}[1]{\textcolor[rgb]{0.73,0.40,0.53}{{#1}}}
    \newcommand{\ImportTok}[1]{{#1}}
    \newcommand{\DocumentationTok}[1]{\textcolor[rgb]{0.73,0.13,0.13}{\textit{{#1}}}}
    \newcommand{\AnnotationTok}[1]{\textcolor[rgb]{0.38,0.63,0.69}{\textbf{\textit{{#1}}}}}
    \newcommand{\CommentVarTok}[1]{\textcolor[rgb]{0.38,0.63,0.69}{\textbf{\textit{{#1}}}}}
    \newcommand{\VariableTok}[1]{\textcolor[rgb]{0.10,0.09,0.49}{{#1}}}
    \newcommand{\ControlFlowTok}[1]{\textcolor[rgb]{0.00,0.44,0.13}{\textbf{{#1}}}}
    \newcommand{\OperatorTok}[1]{\textcolor[rgb]{0.40,0.40,0.40}{{#1}}}
    \newcommand{\BuiltInTok}[1]{{#1}}
    \newcommand{\ExtensionTok}[1]{{#1}}
    \newcommand{\PreprocessorTok}[1]{\textcolor[rgb]{0.74,0.48,0.00}{{#1}}}
    \newcommand{\AttributeTok}[1]{\textcolor[rgb]{0.49,0.56,0.16}{{#1}}}
    \newcommand{\InformationTok}[1]{\textcolor[rgb]{0.38,0.63,0.69}{\textbf{\textit{{#1}}}}}
    \newcommand{\WarningTok}[1]{\textcolor[rgb]{0.38,0.63,0.69}{\textbf{\textit{{#1}}}}}
    
    
    % Define a nice break command that doesn't care if a line doesn't already
    % exist.
    \def\br{\hspace*{\fill} \\* }
    % Math Jax compatability definitions
    \def\gt{>}
    \def\lt{<}
    % Document parameters
    \title{TP3}
    
    
    

    % Pygments definitions
    
\makeatletter
\def\PY@reset{\let\PY@it=\relax \let\PY@bf=\relax%
    \let\PY@ul=\relax \let\PY@tc=\relax%
    \let\PY@bc=\relax \let\PY@ff=\relax}
\def\PY@tok#1{\csname PY@tok@#1\endcsname}
\def\PY@toks#1+{\ifx\relax#1\empty\else%
    \PY@tok{#1}\expandafter\PY@toks\fi}
\def\PY@do#1{\PY@bc{\PY@tc{\PY@ul{%
    \PY@it{\PY@bf{\PY@ff{#1}}}}}}}
\def\PY#1#2{\PY@reset\PY@toks#1+\relax+\PY@do{#2}}

\expandafter\def\csname PY@tok@w\endcsname{\def\PY@tc##1{\textcolor[rgb]{0.73,0.73,0.73}{##1}}}
\expandafter\def\csname PY@tok@c\endcsname{\let\PY@it=\textit\def\PY@tc##1{\textcolor[rgb]{0.25,0.50,0.50}{##1}}}
\expandafter\def\csname PY@tok@cp\endcsname{\def\PY@tc##1{\textcolor[rgb]{0.74,0.48,0.00}{##1}}}
\expandafter\def\csname PY@tok@k\endcsname{\let\PY@bf=\textbf\def\PY@tc##1{\textcolor[rgb]{0.00,0.50,0.00}{##1}}}
\expandafter\def\csname PY@tok@kp\endcsname{\def\PY@tc##1{\textcolor[rgb]{0.00,0.50,0.00}{##1}}}
\expandafter\def\csname PY@tok@kt\endcsname{\def\PY@tc##1{\textcolor[rgb]{0.69,0.00,0.25}{##1}}}
\expandafter\def\csname PY@tok@o\endcsname{\def\PY@tc##1{\textcolor[rgb]{0.40,0.40,0.40}{##1}}}
\expandafter\def\csname PY@tok@ow\endcsname{\let\PY@bf=\textbf\def\PY@tc##1{\textcolor[rgb]{0.67,0.13,1.00}{##1}}}
\expandafter\def\csname PY@tok@nb\endcsname{\def\PY@tc##1{\textcolor[rgb]{0.00,0.50,0.00}{##1}}}
\expandafter\def\csname PY@tok@nf\endcsname{\def\PY@tc##1{\textcolor[rgb]{0.00,0.00,1.00}{##1}}}
\expandafter\def\csname PY@tok@nc\endcsname{\let\PY@bf=\textbf\def\PY@tc##1{\textcolor[rgb]{0.00,0.00,1.00}{##1}}}
\expandafter\def\csname PY@tok@nn\endcsname{\let\PY@bf=\textbf\def\PY@tc##1{\textcolor[rgb]{0.00,0.00,1.00}{##1}}}
\expandafter\def\csname PY@tok@ne\endcsname{\let\PY@bf=\textbf\def\PY@tc##1{\textcolor[rgb]{0.82,0.25,0.23}{##1}}}
\expandafter\def\csname PY@tok@nv\endcsname{\def\PY@tc##1{\textcolor[rgb]{0.10,0.09,0.49}{##1}}}
\expandafter\def\csname PY@tok@no\endcsname{\def\PY@tc##1{\textcolor[rgb]{0.53,0.00,0.00}{##1}}}
\expandafter\def\csname PY@tok@nl\endcsname{\def\PY@tc##1{\textcolor[rgb]{0.63,0.63,0.00}{##1}}}
\expandafter\def\csname PY@tok@ni\endcsname{\let\PY@bf=\textbf\def\PY@tc##1{\textcolor[rgb]{0.60,0.60,0.60}{##1}}}
\expandafter\def\csname PY@tok@na\endcsname{\def\PY@tc##1{\textcolor[rgb]{0.49,0.56,0.16}{##1}}}
\expandafter\def\csname PY@tok@nt\endcsname{\let\PY@bf=\textbf\def\PY@tc##1{\textcolor[rgb]{0.00,0.50,0.00}{##1}}}
\expandafter\def\csname PY@tok@nd\endcsname{\def\PY@tc##1{\textcolor[rgb]{0.67,0.13,1.00}{##1}}}
\expandafter\def\csname PY@tok@s\endcsname{\def\PY@tc##1{\textcolor[rgb]{0.73,0.13,0.13}{##1}}}
\expandafter\def\csname PY@tok@sd\endcsname{\let\PY@it=\textit\def\PY@tc##1{\textcolor[rgb]{0.73,0.13,0.13}{##1}}}
\expandafter\def\csname PY@tok@si\endcsname{\let\PY@bf=\textbf\def\PY@tc##1{\textcolor[rgb]{0.73,0.40,0.53}{##1}}}
\expandafter\def\csname PY@tok@se\endcsname{\let\PY@bf=\textbf\def\PY@tc##1{\textcolor[rgb]{0.73,0.40,0.13}{##1}}}
\expandafter\def\csname PY@tok@sr\endcsname{\def\PY@tc##1{\textcolor[rgb]{0.73,0.40,0.53}{##1}}}
\expandafter\def\csname PY@tok@ss\endcsname{\def\PY@tc##1{\textcolor[rgb]{0.10,0.09,0.49}{##1}}}
\expandafter\def\csname PY@tok@sx\endcsname{\def\PY@tc##1{\textcolor[rgb]{0.00,0.50,0.00}{##1}}}
\expandafter\def\csname PY@tok@m\endcsname{\def\PY@tc##1{\textcolor[rgb]{0.40,0.40,0.40}{##1}}}
\expandafter\def\csname PY@tok@gh\endcsname{\let\PY@bf=\textbf\def\PY@tc##1{\textcolor[rgb]{0.00,0.00,0.50}{##1}}}
\expandafter\def\csname PY@tok@gu\endcsname{\let\PY@bf=\textbf\def\PY@tc##1{\textcolor[rgb]{0.50,0.00,0.50}{##1}}}
\expandafter\def\csname PY@tok@gd\endcsname{\def\PY@tc##1{\textcolor[rgb]{0.63,0.00,0.00}{##1}}}
\expandafter\def\csname PY@tok@gi\endcsname{\def\PY@tc##1{\textcolor[rgb]{0.00,0.63,0.00}{##1}}}
\expandafter\def\csname PY@tok@gr\endcsname{\def\PY@tc##1{\textcolor[rgb]{1.00,0.00,0.00}{##1}}}
\expandafter\def\csname PY@tok@ge\endcsname{\let\PY@it=\textit}
\expandafter\def\csname PY@tok@gs\endcsname{\let\PY@bf=\textbf}
\expandafter\def\csname PY@tok@gp\endcsname{\let\PY@bf=\textbf\def\PY@tc##1{\textcolor[rgb]{0.00,0.00,0.50}{##1}}}
\expandafter\def\csname PY@tok@go\endcsname{\def\PY@tc##1{\textcolor[rgb]{0.53,0.53,0.53}{##1}}}
\expandafter\def\csname PY@tok@gt\endcsname{\def\PY@tc##1{\textcolor[rgb]{0.00,0.27,0.87}{##1}}}
\expandafter\def\csname PY@tok@err\endcsname{\def\PY@bc##1{\setlength{\fboxsep}{0pt}\fcolorbox[rgb]{1.00,0.00,0.00}{1,1,1}{\strut ##1}}}
\expandafter\def\csname PY@tok@kc\endcsname{\let\PY@bf=\textbf\def\PY@tc##1{\textcolor[rgb]{0.00,0.50,0.00}{##1}}}
\expandafter\def\csname PY@tok@kd\endcsname{\let\PY@bf=\textbf\def\PY@tc##1{\textcolor[rgb]{0.00,0.50,0.00}{##1}}}
\expandafter\def\csname PY@tok@kn\endcsname{\let\PY@bf=\textbf\def\PY@tc##1{\textcolor[rgb]{0.00,0.50,0.00}{##1}}}
\expandafter\def\csname PY@tok@kr\endcsname{\let\PY@bf=\textbf\def\PY@tc##1{\textcolor[rgb]{0.00,0.50,0.00}{##1}}}
\expandafter\def\csname PY@tok@bp\endcsname{\def\PY@tc##1{\textcolor[rgb]{0.00,0.50,0.00}{##1}}}
\expandafter\def\csname PY@tok@fm\endcsname{\def\PY@tc##1{\textcolor[rgb]{0.00,0.00,1.00}{##1}}}
\expandafter\def\csname PY@tok@vc\endcsname{\def\PY@tc##1{\textcolor[rgb]{0.10,0.09,0.49}{##1}}}
\expandafter\def\csname PY@tok@vg\endcsname{\def\PY@tc##1{\textcolor[rgb]{0.10,0.09,0.49}{##1}}}
\expandafter\def\csname PY@tok@vi\endcsname{\def\PY@tc##1{\textcolor[rgb]{0.10,0.09,0.49}{##1}}}
\expandafter\def\csname PY@tok@vm\endcsname{\def\PY@tc##1{\textcolor[rgb]{0.10,0.09,0.49}{##1}}}
\expandafter\def\csname PY@tok@sa\endcsname{\def\PY@tc##1{\textcolor[rgb]{0.73,0.13,0.13}{##1}}}
\expandafter\def\csname PY@tok@sb\endcsname{\def\PY@tc##1{\textcolor[rgb]{0.73,0.13,0.13}{##1}}}
\expandafter\def\csname PY@tok@sc\endcsname{\def\PY@tc##1{\textcolor[rgb]{0.73,0.13,0.13}{##1}}}
\expandafter\def\csname PY@tok@dl\endcsname{\def\PY@tc##1{\textcolor[rgb]{0.73,0.13,0.13}{##1}}}
\expandafter\def\csname PY@tok@s2\endcsname{\def\PY@tc##1{\textcolor[rgb]{0.73,0.13,0.13}{##1}}}
\expandafter\def\csname PY@tok@sh\endcsname{\def\PY@tc##1{\textcolor[rgb]{0.73,0.13,0.13}{##1}}}
\expandafter\def\csname PY@tok@s1\endcsname{\def\PY@tc##1{\textcolor[rgb]{0.73,0.13,0.13}{##1}}}
\expandafter\def\csname PY@tok@mb\endcsname{\def\PY@tc##1{\textcolor[rgb]{0.40,0.40,0.40}{##1}}}
\expandafter\def\csname PY@tok@mf\endcsname{\def\PY@tc##1{\textcolor[rgb]{0.40,0.40,0.40}{##1}}}
\expandafter\def\csname PY@tok@mh\endcsname{\def\PY@tc##1{\textcolor[rgb]{0.40,0.40,0.40}{##1}}}
\expandafter\def\csname PY@tok@mi\endcsname{\def\PY@tc##1{\textcolor[rgb]{0.40,0.40,0.40}{##1}}}
\expandafter\def\csname PY@tok@il\endcsname{\def\PY@tc##1{\textcolor[rgb]{0.40,0.40,0.40}{##1}}}
\expandafter\def\csname PY@tok@mo\endcsname{\def\PY@tc##1{\textcolor[rgb]{0.40,0.40,0.40}{##1}}}
\expandafter\def\csname PY@tok@ch\endcsname{\let\PY@it=\textit\def\PY@tc##1{\textcolor[rgb]{0.25,0.50,0.50}{##1}}}
\expandafter\def\csname PY@tok@cm\endcsname{\let\PY@it=\textit\def\PY@tc##1{\textcolor[rgb]{0.25,0.50,0.50}{##1}}}
\expandafter\def\csname PY@tok@cpf\endcsname{\let\PY@it=\textit\def\PY@tc##1{\textcolor[rgb]{0.25,0.50,0.50}{##1}}}
\expandafter\def\csname PY@tok@c1\endcsname{\let\PY@it=\textit\def\PY@tc##1{\textcolor[rgb]{0.25,0.50,0.50}{##1}}}
\expandafter\def\csname PY@tok@cs\endcsname{\let\PY@it=\textit\def\PY@tc##1{\textcolor[rgb]{0.25,0.50,0.50}{##1}}}

\def\PYZbs{\char`\\}
\def\PYZus{\char`\_}
\def\PYZob{\char`\{}
\def\PYZcb{\char`\}}
\def\PYZca{\char`\^}
\def\PYZam{\char`\&}
\def\PYZlt{\char`\<}
\def\PYZgt{\char`\>}
\def\PYZsh{\char`\#}
\def\PYZpc{\char`\%}
\def\PYZdl{\char`\$}
\def\PYZhy{\char`\-}
\def\PYZsq{\char`\'}
\def\PYZdq{\char`\"}
\def\PYZti{\char`\~}
% for compatibility with earlier versions
\def\PYZat{@}
\def\PYZlb{[}
\def\PYZrb{]}
\makeatother


    % Exact colors from NB
    \definecolor{incolor}{rgb}{0.0, 0.0, 0.5}
    \definecolor{outcolor}{rgb}{0.545, 0.0, 0.0}



    
    % Prevent overflowing lines due to hard-to-break entities
    \sloppy 
    % Setup hyperref package
    \hypersetup{
      breaklinks=true,  % so long urls are correctly broken across lines
      colorlinks=true,
      urlcolor=urlcolor,
      linkcolor=linkcolor,
      citecolor=citecolor,
      }
    % Slightly bigger margins than the latex defaults
    
    \geometry{verbose,tmargin=1in,bmargin=1in,lmargin=1in,rmargin=1in}
    
    

    \begin{document}
    
    
    \maketitle
    
    

    
    \section{SD-TSIA214 Machine Learning For Text
Mining}\label{sd-tsia214-machine-learning-for-text-mining}

\section{Text segmentation using Hidden Markov
Models}\label{text-segmentation-using-hidden-markov-models}

    \begin{Verbatim}[commandchars=\\\{\}]
{\color{incolor}In [{\color{incolor}1}]:} \PY{k+kn}{import} \PY{n+nn}{numpy} \PY{k}{as} \PY{n+nn}{np}
\end{Verbatim}


    \subsection{Task : automatic segmentation of
mails}\label{task-automatic-segmentation-of-mails}

This Lab aims to build an email segmentation tool, dedicated to separate
the email header from its body. It is proposed to perform this task by
learning a HMM with two states, one (state 1) for the header, the other
(state 2) for the body. In this model, it is assumed that each mail
actually contains a header : the decoding necessarily begins in the
state 1.

\subsection{1. Give the value of the π vector of the initial
probabilities}\label{give-the-value-of-the-ux3c0-vector-of-the-initial-probabilities}

It is said above that our HMM has necessarily an initial state of 1. So,
the π vector of the initial probabilities will have the following form:

\[\pi^{T} = (1, 0)\]

Since the HMM will be in state 0 with probability 0 and in state 1 with
probability 1, i.e. always.

Knowing that each mail contains exactly one header and one body, each
mail follows once the transition from 1 to 2. The transition matrix
(A(i, j) = P(j\textbar{}i)) estimated on a labeled small corpus has thus
the following form :

    \begin{Verbatim}[commandchars=\\\{\}]
{\color{incolor}In [{\color{incolor}2}]:} \PY{n}{A} \PY{o}{=} \PY{n}{np}\PY{o}{.}\PY{n}{array}\PY{p}{(}\PY{p}{[}\PY{p}{[}\PY{l+m+mf}{0.999218078035812}\PY{p}{,} \PY{l+m+mf}{0.000781921964187974}\PY{p}{]}\PY{p}{,} \PY{p}{[}\PY{l+m+mi}{0}\PY{p}{,} \PY{l+m+mi}{1}\PY{p}{]}\PY{p}{]}\PY{p}{)}
        \PY{n}{A}
\end{Verbatim}


\begin{Verbatim}[commandchars=\\\{\}]
{\color{outcolor}Out[{\color{outcolor}2}]:} array([[9.99218078e-01, 7.81921964e-04],
               [0.00000000e+00, 1.00000000e+00]])
\end{Verbatim}
            
    \subsection{2. What is the probability to move from state 1 to state 2 ?
What is the probability to remain in state 2 ? What is the lower/higher
probability ? Try to explain
why}\label{what-is-the-probability-to-move-from-state-1-to-state-2-what-is-the-probability-to-remain-in-state-2-what-is-the-lowerhigher-probability-try-to-explain-why}

Probabilities of movements between states can be found in the transition
matrix A, which was just given. The i index (row index) determines the
starting state, while the j index (column index) determines the arriving
state. Thus:

\begin{itemize}
\tightlist
\item
  Probability to move from state 1 to state 2: A(1,2) =
\end{itemize}

    \begin{Verbatim}[commandchars=\\\{\}]
{\color{incolor}In [{\color{incolor}3}]:} \PY{n}{A}\PY{p}{[}\PY{l+m+mi}{0}\PY{p}{,}\PY{l+m+mi}{1}\PY{p}{]}
\end{Verbatim}


\begin{Verbatim}[commandchars=\\\{\}]
{\color{outcolor}Out[{\color{outcolor}3}]:} 0.000781921964187974
\end{Verbatim}
            
    \begin{itemize}
\tightlist
\item
  Probability to remain in state 2: A(2,2) =
\end{itemize}

    \begin{Verbatim}[commandchars=\\\{\}]
{\color{incolor}In [{\color{incolor}4}]:} \PY{n}{A}\PY{p}{[}\PY{l+m+mi}{1}\PY{p}{,}\PY{l+m+mi}{1}\PY{p}{]}
\end{Verbatim}


\begin{Verbatim}[commandchars=\\\{\}]
{\color{outcolor}Out[{\color{outcolor}4}]:} 1.0
\end{Verbatim}
            
    Probability to remain in state 2 is actually the highest obtainable
probability (=1), and this makes sense in our use case: once our email
is in the state corresponding to the body, we can't go back anymore to
the header state with more iterations, that correspond to reading more
characters. In fact, an email's body goes from its beginning to end of
an email, and in no case is followed by another header. As a
consequence, when the number of iterations/observation n goes to
infinity, our observed Markov Chain will forcely be in state 1, meaning
that - as stated before - we will necessarily be in the email's body.

On the contrary, the probability of moving from state 1 to state 2 is
much lower because in average we have to read a significant number of
characters - and, therefore, wait for several iterations - before being
able to affirm that the email has also a body.

For example, if we had built our model and the transition matrix A using
some strange kind of emails that have 3 characters in the header and (in
average) more than 2000 in the body, this probability would be much
higher. Obviously, the creation of the HMM and thus our probabilty also
depend on the observations, which influence directly the hidden process
and indirectly the observed one: if I had trained it with headers
containing only 'a' characters and bodies that never contain the 'a'
character, this would surely impact the transition probability matrix.

A mail is represented by a sequence of characters. Let N be the number
of different characters. Each part of the mail is characterized by a
discrete probability distribution on the characters P(c\textbar{}s),
with s = 1 or s = 2.

\subsection{3. What is the size of the corresponding matrix
?}\label{what-is-the-size-of-the-corresponding-matrix}

The size of the corrispondig matrix is \[ N \times |s| \] Since we are
encoding characters as ASCII values, which range from 1 to 256, the
matrix will have a size of 256x2, namely 256 rows and 2 columns.

\subsection{Material}\label{material}

\subsubsection{Coding/decoding mails}\label{codingdecoding-mails}

Emails are represented as ASCII character vectors. In dat.zip, mail.txt
in a vector of numbers (one vector per line). that can be transformed
into a vector of numbers (between 0 and 255) in a text; Files of the
form dat /*. dat contain the already encoded versions of the
corresponding mails. The list is in mail.lst.

\subsubsection{Visualizing segmentation}\label{visualizing-segmentation}

Finally, the utility segment.pl allows to visualize a segmentation
produced by your segmenter in the form of the best path found by the
Viterbi algorithm (in a vector of 1 and 2). It produces a file where the
segmentation is visualized.

To use it :

perl segment.pl mail.txt path.dat

\subsubsection{Distribution files}\label{distribution-files}

For the first part of the Lab, we work with the distributions that are
provided in the P.dat file.

Each of the columns of this file contains the distribution of the
probabilities of occurrence of each character of the ASCII codes
respectively in the header and in the body. These distributions were
learned on a small corpus labeled with 10 emails ; there are obvious
differences, especially in areas where ASCII codes correspond to
alphabetic characters, as you can see by viewing these distributions.

\paragraph{To implement:}\label{to-implement}

All the work is to be done under Python.

\begin{itemize}
\tightlist
\item
  implement the Viterbi algorithm. Concretely, it comes to coding a
  function which takes as argument a vector of observations and the
  parameters of the model, and returns a vector of states representing
  the most probable sequence.
\end{itemize}

    \begin{Verbatim}[commandchars=\\\{\}]
{\color{incolor}In [{\color{incolor}5}]:} \PY{l+s+sd}{\PYZdq{}\PYZdq{}\PYZdq{}}
        \PY{l+s+sd}{    Viterbi Algorithm Implementation}
        \PY{l+s+sd}{    }
        \PY{l+s+sd}{    Inputs:}
        \PY{l+s+sd}{        \PYZhy{} obs: a 1\PYZhy{}d nparray containing the observations. Its values must be lesser }
        \PY{l+s+sd}{            or equal than the number of rows of the conditional probability matrix. }
        \PY{l+s+sd}{        \PYZhy{} states: a 1\PYZhy{}d nparray containing the number of possible states. Its size}
        \PY{l+s+sd}{            has to be equal to the number of columns of the conditional probability }
        \PY{l+s+sd}{            matrix.}
        \PY{l+s+sd}{        \PYZhy{} trans: a 2\PYZhy{}d nparray transition probability matrix containing the }
        \PY{l+s+sd}{            transition probabilities for the observed process.}
        \PY{l+s+sd}{        \PYZhy{} start\PYZus{}prob: a 1\PYZhy{}d nparray containing the starting probabilities for the}
        \PY{l+s+sd}{            observed process. Its length has to be equal to the number of possible}
        \PY{l+s+sd}{            states.}
        \PY{l+s+sd}{        \PYZhy{} cond\PYZus{}prob: also known as emission probability, it is a 2\PYZhy{}d nparray }
        \PY{l+s+sd}{            having a number of columns equal to the number of possible states.}
        \PY{l+s+sd}{    }
        \PY{l+s+sd}{    Returns:}
        \PY{l+s+sd}{        \PYZhy{} seq: a 1\PYZhy{}d nparray containing, for each observation given in the input,}
        \PY{l+s+sd}{            the corresponding state that represent the most probable sequence}
        \PY{l+s+sd}{            }
        \PY{l+s+sd}{\PYZdq{}\PYZdq{}\PYZdq{}}
        
        \PY{k}{def} \PY{n+nf}{viterbiAlgorithm}\PY{p}{(}\PY{n}{obs}\PY{p}{,} \PY{n}{states}\PY{p}{,} \PY{n}{trans}\PY{p}{,} \PY{n}{start\PYZus{}prob}\PY{p}{,} \PY{n}{cond\PYZus{}prob}\PY{p}{)}\PY{p}{:}
            \PY{c+c1}{\PYZsh{} Rescaling function that avoids state probabilities going to zero}
            \PY{c+c1}{\PYZsh{} Computes the order of magnitude of values in the array\PYZsq{}s last row }
            \PY{c+c1}{\PYZsh{} and rescales them back to 10\PYZca{}\PYZhy{}1}
            \PY{k}{def} \PY{n+nf}{rescale}\PY{p}{(}\PY{n}{state\PYZus{}probs}\PY{p}{,} \PY{n}{last\PYZus{}row}\PY{p}{)}\PY{p}{:}
                \PY{c+c1}{\PYZsh{} Compute order of magnitude}
                \PY{n}{om} \PY{o}{=} \PY{n+nb}{int}\PY{p}{(}\PY{n}{np}\PY{o}{.}\PY{n}{log10}\PY{p}{(}\PY{n}{np}\PY{o}{.}\PY{n}{max}\PY{p}{(}\PY{n}{state\PYZus{}probs}\PY{p}{[}\PY{n}{last\PYZus{}row}\PY{p}{,}\PY{p}{:}\PY{p}{]}\PY{p}{)}\PY{p}{)}\PY{p}{)}
                \PY{c+c1}{\PYZsh{} Rescale}
        \PY{c+c1}{\PYZsh{}         print(\PYZsq{}Before rescaling \PYZsq{} + str(state\PYZus{}probs[last\PYZus{}row, :]) }
        \PY{c+c1}{\PYZsh{}                    + \PYZsq{} \PYZhy{} OM \PYZsq{} + str(om) + \PYZsq{} Resc factor \PYZsq{} + str((om+1)))}
                \PY{n}{state\PYZus{}probs}\PY{p}{[}\PY{n}{last\PYZus{}row}\PY{p}{,} \PY{p}{:}\PY{p}{]} \PY{o}{=} \PY{n}{state\PYZus{}probs}\PY{p}{[}\PY{n}{last\PYZus{}row}\PY{p}{,} \PY{p}{:}\PY{p}{]}\PY{o}{*}\PY{p}{(}\PY{l+m+mi}{10}\PY{o}{*}\PY{o}{*}\PY{p}{(}\PY{o}{\PYZhy{}}\PY{l+m+mi}{1}\PY{o}{*}\PY{p}{(}\PY{n}{om}\PY{o}{+}\PY{l+m+mi}{1}\PY{p}{)}\PY{p}{)}\PY{p}{)}
        \PY{c+c1}{\PYZsh{}         print(\PYZsq{}After rescaling \PYZsq{} + str(state\PYZus{}probs[last\PYZus{}row, :]))}
                \PY{k}{return} \PY{n}{state\PYZus{}probs}
            
            \PY{c+c1}{\PYZsh{} State probabilities}
            \PY{n}{state\PYZus{}probs} \PY{o}{=} \PY{n}{np}\PY{o}{.}\PY{n}{zeros}\PY{p}{(}\PY{p}{(}\PY{n+nb}{len}\PY{p}{(}\PY{n}{obs}\PY{p}{)}\PY{p}{,} \PY{n+nb}{len}\PY{p}{(}\PY{n}{states}\PY{p}{)}\PY{p}{)}\PY{p}{)}
            \PY{c+c1}{\PYZsh{} Sequence of states}
            \PY{n}{path} \PY{o}{=} \PY{p}{\PYZob{}}\PY{p}{\PYZcb{}}
            
            \PY{k}{for} \PY{n}{i} \PY{o+ow}{in} \PY{n+nb}{range}\PY{p}{(}\PY{l+m+mi}{0}\PY{p}{,}\PY{n+nb}{len}\PY{p}{(}\PY{n}{observations}\PY{p}{)}\PY{p}{)}\PY{p}{:}
                
                \PY{c+c1}{\PYZsh{} First iteration}
                \PY{k}{if} \PY{n}{i} \PY{o}{==} \PY{l+m+mi}{0}\PY{p}{:}
                    \PY{c+c1}{\PYZsh{} Initialize states probabilites }
                    \PY{k}{for} \PY{n}{state} \PY{o+ow}{in} \PY{n+nb}{range}\PY{p}{(}\PY{l+m+mi}{0}\PY{p}{,}\PY{n+nb}{len}\PY{p}{(}\PY{n}{states}\PY{p}{)}\PY{p}{)}\PY{p}{:}
                        \PY{n}{state\PYZus{}probs}\PY{p}{[}\PY{n}{i}\PY{p}{]}\PY{p}{[}\PY{n}{state}\PY{p}{]} \PY{o}{=} \PY{n}{start\PYZus{}prob}\PY{p}{[}\PY{n}{state}\PY{p}{]}\PYZbs{}
                                                    \PY{o}{*} \PY{n}{cond\PYZus{}prob}\PY{p}{[}\PY{n}{obs}\PY{p}{[}\PY{n}{i}\PY{p}{]}\PY{p}{]}\PY{p}{[}\PY{n}{state}\PY{p}{]}
                        \PY{n}{path}\PY{p}{[}\PY{n}{state}\PY{p}{]} \PY{o}{=} \PY{p}{[}\PY{n}{states}\PY{p}{[}\PY{n}{state}\PY{p}{]}\PY{p}{]}
        \PY{c+c1}{\PYZsh{}             print(\PYZsq{}States probabilities: \PYZsq{} + str(state\PYZus{}probs[i]))}
                \PY{k}{else}\PY{p}{:}
                    \PY{c+c1}{\PYZsh{} Update}
                    \PY{n}{tempPath} \PY{o}{=} \PY{p}{\PYZob{}}\PY{p}{\PYZcb{}}
                    \PY{n}{probs} \PY{o}{=} \PY{n}{np}\PY{o}{.}\PY{n}{zeros}\PY{p}{(}\PY{p}{(}\PY{n+nb}{len}\PY{p}{(}\PY{n}{states}\PY{p}{)}\PY{p}{,} \PY{n+nb}{len}\PY{p}{(}\PY{n}{states}\PY{p}{)}\PY{p}{)}\PY{p}{)}            
                    \PY{k}{for} \PY{n}{arr\PYZus{}state} \PY{o+ow}{in} \PY{n+nb}{range}\PY{p}{(}\PY{l+m+mi}{0}\PY{p}{,} \PY{n+nb}{len}\PY{p}{(}\PY{n}{states}\PY{p}{)}\PY{p}{)}\PY{p}{:}
                        \PY{p}{(}\PY{n}{probs}\PY{p}{,} \PY{n}{state}\PY{p}{)} \PY{o}{=} \PY{n+nb}{max}\PY{p}{(}\PY{p}{[}\PY{p}{(}\PY{n}{state\PYZus{}probs}\PY{p}{[}\PY{n}{i}\PY{o}{\PYZhy{}}\PY{l+m+mi}{1}\PY{p}{]}\PY{p}{[}\PY{n}{dep\PYZus{}state}\PY{p}{]} \PYZbs{}
                                               \PY{o}{*} \PY{n}{trans}\PY{p}{[}\PY{n}{dep\PYZus{}state}\PY{p}{]}\PY{p}{[}\PY{n}{arr\PYZus{}state}\PY{p}{]}\PYZbs{}
                                               \PY{o}{*} \PY{n}{cond\PYZus{}prob}\PY{p}{[}\PY{n}{obs}\PY{p}{[}\PY{n}{i}\PY{p}{]}\PY{p}{]}\PY{p}{[}\PY{n}{arr\PYZus{}state}\PY{p}{]}\PY{p}{,} 
                                              \PY{n}{dep\PYZus{}state}\PY{p}{)} \PYZbs{}
                                              \PY{k}{for} \PY{n}{dep\PYZus{}state} \PY{o+ow}{in} \PY{n+nb}{range}\PY{p}{(}\PY{l+m+mi}{0}\PY{p}{,} \PY{n+nb}{len}\PY{p}{(}\PY{n}{states}\PY{p}{)}\PY{p}{)}\PY{p}{]}\PY{p}{)}
        \PY{c+c1}{\PYZsh{}               print(\PYZsq{}Arrival \PYZsq{} + str(arr\PYZus{}state) + \PYZsq{} \PYZhy{} Probabilities : \PYZsq{}}
        \PY{c+c1}{\PYZsh{}               + str((probs, state)))}
                        \PY{n}{state\PYZus{}probs}\PY{p}{[}\PY{n}{i}\PY{p}{,} \PY{n}{arr\PYZus{}state}\PY{p}{]} \PY{o}{=} \PY{n}{probs}
                        \PY{n}{tempPath}\PY{p}{[}\PY{n}{arr\PYZus{}state}\PY{p}{]} \PY{o}{=} \PY{n}{path}\PY{p}{[}\PY{n}{state}\PY{p}{]} \PY{o}{+} \PY{p}{[}\PY{n}{states}\PY{p}{[}\PY{n}{arr\PYZus{}state}\PY{p}{]}\PY{p}{]}
        \PY{c+c1}{\PYZsh{}             print(\PYZsq{}Observation \PYZsq{} + str(i) + \PYZsq{} \PYZhy{} Probabilities: \PYZsq{})}
        \PY{c+c1}{\PYZsh{}             print(str(probs))}
        \PY{c+c1}{\PYZsh{}             print(\PYZsq{}Max prob : \PYZsq{} + str(probs[r][c]) + \PYZsq{} \PYZhy{} R, C : \PYZsq{} + str(r) + \PYZsq{}, \PYZsq{}}
        \PY{c+c1}{\PYZsh{}             + str(c) + \PYZsq{} \PYZhy{} State: \PYZsq{} + str(states[c]))}
        \PY{c+c1}{\PYZsh{}             print(\PYZsq{}States probabilities: \PYZsq{} + str(state\PYZus{}probs[i]))}
                    \PY{n}{path} \PY{o}{=}  \PY{n}{tempPath}   
                    \PY{n}{state\PYZus{}probs} \PY{o}{=} \PY{n}{rescale}\PY{p}{(}\PY{n}{state\PYZus{}probs}\PY{p}{,} \PY{n}{i}\PY{p}{)}
            \PY{p}{(}\PY{n}{finProb}\PY{p}{,} \PY{n}{state}\PY{p}{)} \PY{o}{=} \PY{n+nb}{max}\PY{p}{(}\PY{p}{[}\PY{p}{(}\PY{n}{state\PYZus{}probs}\PY{p}{[}\PY{o}{\PYZhy{}}\PY{l+m+mi}{1}\PY{p}{]}\PY{p}{[}\PY{n}{s}\PY{p}{]}\PY{p}{,} \PY{n}{s}\PY{p}{)}\PYZbs{}
                                    \PY{k}{for} \PY{n}{s} \PY{o+ow}{in} \PY{n+nb}{range}\PY{p}{(}\PY{l+m+mi}{0}\PY{p}{,} \PY{n+nb}{len}\PY{p}{(}\PY{n}{states}\PY{p}{)}\PY{p}{)}\PY{p}{]}\PY{p}{)}
            \PY{k}{return} \PY{n}{path}\PY{p}{[}\PY{n}{state}\PY{p}{]}
\end{Verbatim}


    \begin{itemize}
\tightlist
\item
  test it on some mails that are given in the dat directory (especially
  mail11.txt to mail30.txt).
\end{itemize}

    \begin{Verbatim}[commandchars=\\\{\}]
{\color{incolor}In [{\color{incolor}6}]:} \PY{k}{def} \PY{n+nf}{printPath}\PY{p}{(}\PY{n}{path}\PY{p}{)}\PY{p}{:}
            \PY{n}{summary} \PY{o}{=} \PY{p}{[}\PY{p}{]}
            \PY{n}{summary}\PY{o}{.}\PY{n}{append}\PY{p}{(}\PY{p}{\PYZob{}} \PY{l+s+s1}{\PYZsq{}}\PY{l+s+s1}{start}\PY{l+s+s1}{\PYZsq{}}\PY{p}{:} \PY{l+m+mi}{0}\PY{p}{,} \PY{l+s+s1}{\PYZsq{}}\PY{l+s+s1}{end}\PY{l+s+s1}{\PYZsq{}}\PY{p}{:} \PY{l+m+mi}{0}\PY{p}{,} \PY{l+s+s1}{\PYZsq{}}\PY{l+s+s1}{value}\PY{l+s+s1}{\PYZsq{}}\PY{p}{:} \PY{n}{path}\PY{p}{[}\PY{l+m+mi}{0}\PY{p}{]}\PY{p}{\PYZcb{}}\PY{p}{)}
            \PY{k}{for} \PY{n}{i} \PY{o+ow}{in} \PY{n+nb}{range}\PY{p}{(}\PY{l+m+mi}{1}\PY{p}{,} \PY{n+nb}{len}\PY{p}{(}\PY{n}{path}\PY{p}{)}\PY{p}{)}\PY{p}{:}
                \PY{k}{if} \PY{n}{path}\PY{p}{[}\PY{n}{i}\PY{p}{]} \PY{o}{!=} \PY{n}{path}\PY{p}{[}\PY{n}{i}\PY{o}{\PYZhy{}}\PY{l+m+mi}{1}\PY{p}{]}\PY{p}{:}
                    \PY{n}{summary}\PY{o}{.}\PY{n}{append}\PY{p}{(}\PY{p}{\PYZob{}} \PY{l+s+s1}{\PYZsq{}}\PY{l+s+s1}{start}\PY{l+s+s1}{\PYZsq{}}\PY{p}{:} \PY{n}{i}\PY{p}{,} \PY{l+s+s1}{\PYZsq{}}\PY{l+s+s1}{end}\PY{l+s+s1}{\PYZsq{}}\PY{p}{:} \PY{n}{i}\PY{p}{,} \PY{l+s+s1}{\PYZsq{}}\PY{l+s+s1}{value}\PY{l+s+s1}{\PYZsq{}}\PY{p}{:} \PY{n}{path}\PY{p}{[}\PY{n}{i}\PY{p}{]} \PY{p}{\PYZcb{}}\PY{p}{)}
                \PY{k}{else}\PY{p}{:}
                    \PY{n}{summary}\PY{p}{[}\PY{o}{\PYZhy{}}\PY{l+m+mi}{1}\PY{p}{]}\PY{p}{[}\PY{l+s+s1}{\PYZsq{}}\PY{l+s+s1}{end}\PY{l+s+s1}{\PYZsq{}}\PY{p}{]} \PY{o}{=} \PY{n}{i}
            
            \PY{k}{for} \PY{n}{s} \PY{o+ow}{in} \PY{n}{summary}\PY{p}{:}
                \PY{n+nb}{print}\PY{p}{(}\PY{l+s+s1}{\PYZsq{}}\PY{l+s+s1}{State }\PY{l+s+s1}{\PYZsq{}} \PY{o}{+} \PY{n+nb}{str}\PY{p}{(}\PY{n}{s}\PY{p}{[}\PY{l+s+s1}{\PYZsq{}}\PY{l+s+s1}{value}\PY{l+s+s1}{\PYZsq{}}\PY{p}{]}\PY{p}{)} \PY{o}{+} \PY{l+s+s1}{\PYZsq{}}\PY{l+s+s1}{ observations }\PY{l+s+s1}{\PYZsq{}} 
                      \PY{o}{+} \PY{n+nb}{str}\PY{p}{(}\PY{n}{s}\PY{p}{[}\PY{l+s+s1}{\PYZsq{}}\PY{l+s+s1}{start}\PY{l+s+s1}{\PYZsq{}}\PY{p}{]}\PY{p}{)} \PY{o}{+} \PY{l+s+s1}{\PYZsq{}}\PY{l+s+s1}{ \PYZhy{}\PYZgt{} }\PY{l+s+s1}{\PYZsq{}} \PY{o}{+} \PY{n+nb}{str}\PY{p}{(}\PY{n}{s}\PY{p}{[}\PY{l+s+s1}{\PYZsq{}}\PY{l+s+s1}{end}\PY{l+s+s1}{\PYZsq{}}\PY{p}{]}\PY{p}{)}\PY{p}{)}
        
        \PY{n}{distributions} \PY{o}{=} \PY{n}{np}\PY{o}{.}\PY{n}{loadtxt}\PY{p}{(}\PY{l+s+s1}{\PYZsq{}}\PY{l+s+s1}{p.text}\PY{l+s+s1}{\PYZsq{}}\PY{p}{,} \PY{n}{dtype}\PY{o}{=}\PY{n+nb}{float}\PY{p}{)}
        
        \PY{k}{for} \PY{n}{i} \PY{o+ow}{in} \PY{n+nb}{range}\PY{p}{(}\PY{l+m+mi}{11}\PY{p}{,}\PY{l+m+mi}{31}\PY{p}{)}\PY{p}{:}
            \PY{n}{observations} \PY{o}{=} \PY{n}{np}\PY{o}{.}\PY{n}{loadtxt}\PY{p}{(}\PY{l+s+s1}{\PYZsq{}}\PY{l+s+s1}{./dat/mail}\PY{l+s+s1}{\PYZsq{}}\PY{o}{+}\PY{n+nb}{str}\PY{p}{(}\PY{n}{i}\PY{p}{)}\PY{o}{+}\PY{l+s+s1}{\PYZsq{}}\PY{l+s+s1}{.dat}\PY{l+s+s1}{\PYZsq{}}\PY{p}{,} \PY{n}{dtype}\PY{o}{=}\PY{n+nb}{int}\PY{p}{)}
            \PY{n}{p} \PY{o}{=} \PY{n}{viterbiAlgorithm}\PY{p}{(}\PY{n}{obs}\PY{o}{=}\PY{n}{observations}\PY{p}{,} \PY{n}{states}\PY{o}{=}\PY{p}{[}\PY{l+m+mi}{1}\PY{p}{,}\PY{l+m+mi}{2}\PY{p}{]}\PY{p}{,} \PY{n}{start\PYZus{}prob}\PY{o}{=}\PY{p}{[}\PY{l+m+mi}{1}\PY{p}{,}\PY{l+m+mi}{0}\PY{p}{]}\PY{p}{,} \PY{n}{trans}\PY{o}{=}\PY{n}{A}\PY{p}{,} \PY{n}{cond\PYZus{}prob}\PY{o}{=}\PY{n}{distributions}\PY{p}{)}
            \PY{n+nb}{print}\PY{p}{(}\PY{l+s+s1}{\PYZsq{}}\PY{l+s+s1}{\PYZhy{}\PYZhy{}\PYZhy{} MAIL }\PY{l+s+s1}{\PYZsq{}} \PY{o}{+} \PY{n+nb}{str}\PY{p}{(}\PY{n}{i}\PY{p}{)} \PY{o}{+} \PY{l+s+s1}{\PYZsq{}}\PY{l+s+s1}{ STATES : }\PY{l+s+s1}{\PYZsq{}}\PY{p}{)}
            \PY{n}{printPath}\PY{p}{(}\PY{n}{p}\PY{p}{)}
\end{Verbatim}


    \begin{Verbatim}[commandchars=\\\{\}]
--- MAIL 11 STATES : 
State 1 observations 0 -> 2850
State 2 observations 2851 -> 3474
--- MAIL 12 STATES : 
State 1 observations 0 -> 2937
State 2 observations 2938 -> 3992
--- MAIL 13 STATES : 
State 1 observations 0 -> 2303
State 2 observations 2304 -> 3327
--- MAIL 14 STATES : 
State 1 observations 0 -> 4812
State 2 observations 4813 -> 6575
--- MAIL 15 STATES : 
State 1 observations 0 -> 2182
State 2 observations 2183 -> 6807
--- MAIL 16 STATES : 
State 1 observations 0 -> 1970
State 2 observations 1971 -> 2626
--- MAIL 17 STATES : 
State 1 observations 0 -> 2281
State 2 observations 2282 -> 3424
--- MAIL 18 STATES : 
State 1 observations 0 -> 2366
State 2 observations 2367 -> 3076
--- MAIL 19 STATES : 
State 1 observations 0 -> 2100
State 2 observations 2101 -> 2619
--- MAIL 20 STATES : 
State 1 observations 0 -> 1839
State 2 observations 1840 -> 2433
--- MAIL 21 STATES : 
State 1 observations 0 -> 2102
State 2 observations 2103 -> 2663
--- MAIL 22 STATES : 
State 1 observations 0 -> 2233
State 2 observations 2234 -> 3642
--- MAIL 23 STATES : 
State 1 observations 0 -> 2167
State 2 observations 2168 -> 3749
--- MAIL 24 STATES : 
State 1 observations 0 -> 2559
State 2 observations 2560 -> 3700
--- MAIL 25 STATES : 
State 1 observations 0 -> 2318
State 2 observations 2319 -> 3237
--- MAIL 26 STATES : 
State 1 observations 0 -> 2026
State 2 observations 2027 -> 4466
--- MAIL 27 STATES : 
State 1 observations 0 -> 1770
State 2 observations 1771 -> 3147
--- MAIL 28 STATES : 
State 1 observations 0 -> 2224
State 2 observations 2225 -> 2540
--- MAIL 29 STATES : 
State 1 observations 0 -> 2343
State 2 observations 2344 -> 2889
--- MAIL 30 STATES : 
State 1 observations 0 -> 2172
State 2 observations 2173 -> 5159

    \end{Verbatim}

    \subsection{4. Print the track and present and discuss the results
obtained on mail11.txt to
mail30.txt}\label{print-the-track-and-present-and-discuss-the-results-obtained-on-mail11.txt-to-mail30.txt}

    The results seem consistent, first of all because of their values: our
HMM starts in State 1, goes to State 2 after a considerable number of
characters and never comes back to State 1.

Furthermore, I checked some emails and results are encouraging. For
example, in mail11 our algorithm states that body starts at character
2850 while I found it beginning at character 2851: it could be my fault
(I was using Atom editor to select characters and I'm not 100\% sure of
where it starts/ends selecting them), but the results is anyhow very
precise!

Obviously, there are some more significant errors, as in email14, where
the model struggles a little to recognize the beginning of the body
because it begins with a quote from a previous email that actually looks
like a header. I think, however, that this is a limitation of the whole
model: if I sent an email containing the header of another email in the
body, the model will fail to distinguish between the real model and the
copied header in the body.

    \subsection{5. How would you model the problem if you had to segment the
mails in more than two parts (for example : header, body, signature)
?}\label{how-would-you-model-the-problem-if-you-had-to-segment-the-mails-in-more-than-two-parts-for-example-header-body-signature}

In this case, our the hidden model wouldn't change since the possible
values of our observations (characters) remain the same: instead, it
would change the observed model which will now have three states (namely
header, body and signature). The transition matrix will hence become a
3x3 matrix of this form:

\[ \begin{bmatrix}
    p_{11}       & p_{12} & 0 \\
     0       & p_{22} & p_{23} \\
     0       & 0 & 1
 \end{bmatrix} \]

and the initial vector π:

\[π^{T} = (1, 0, 0)\]

\subsection{6. How would you model the problem of separating the
portions of mail included, knowing that they always start with the
character
"\textgreater{}".}\label{how-would-you-model-the-problem-of-separating-the-portions-of-mail-included-knowing-that-they-always-start-with-the-character-.}

The model would now have four states, namely header, body\_text,
body\_included, and signature, with a transition matrix 4x4 of this
form:

\[ \begin{bmatrix}
    p_{11}       & p_{12} & p_{13} & 0 \\
     0       & p_{22} & p_{23} & p_{24} \\
     0       & p_{32} & p_{33} & p_{34} \\
     0       & 0 & 0 & 1
 \end{bmatrix} \]

and the initial vector π:

\[\pi^{T} = (1, 0, 0, 0)\]

where body\_text and body\_include are now respectively state 2 and 3,
while signature becomes state 4.

The fact of knowing that included mail always start with a
"\textgreater{}" character is an information that regards only the
outcome of the observation, and would therefore influence not the
observation model but the hidden one: the conditional probability vector
will have a higher probability at the place corresponding to character
"\textgreater{}" and state 3.

More precisely, if the vector of conditional probabilities is built only
by looking if a character is present or not in the desired portion of
speech, this probability would be equal to 1 for the class corresponding
to included mails: since included mails always start with the character
"\textgreater{}", the probability of having a "\textgreater{}"
conditioned to the fact of being in class body\_included is 1.


    % Add a bibliography block to the postdoc
    
    
    
    \end{document}
